\clearpage
\appendix

\input{content/notation}\clearpage

\section{Graph Networks}

\newlength{\threesubht}
\newsavebox{\threesubbox}
\begin{figure}[h]
    \centering
    \sbox\threesubbox{
      \resizebox{\dimexpr.95\textwidth-1em}{!}{
        \includegraphics{resources/gn-stack-config}
        \includegraphics{resources/gn-epd-config}
        \includegraphics{resources/gn-rnn-config}
      }
    }
    \setlength{\threesubht}{\ht\threesubbox}
    \subcaptionbox{Composition of GN blocks\label{fig:gn-composition}}{%
      \includegraphics[height=1\threesubht]{resources/gn-stack-config}%
    }\quad
    \subcaptionbox{Encode-process-decode\label{fig:gn-epd-config}}{%
      \includegraphics[height=1\threesubht]{resources/gn-epd-config}%
    }\quad
    \subcaptionbox{Recurrent GN architecture\label{fig:gn-rnn-config}}{%
      \includegraphics[height=1\threesubht]{resources/gn-rnn-config}%
    }
        \caption{(a) An example composing multiple GN blocks in sequence to form a GN ``core''. Here, the GN blocks can use shared weights, or they could be independent. (b) The \emph{encode-process-decode} architecture, which is a common choice for composing GN blocks. Here, a GN encodes an input graph, which is then processed by a GN core. The output of the core is decoded by a third GN block into an output graph, whose nodes, edges, and/or global attributes would be used for task-specific purposes. (c) The encode-process-decode architecture applied in a sequential setting in which the core is also unrolled over time (potentially using a GRU or LSTM architecture), in addition to being repeated within each time step. Here, merged lines indicate concatenation, and split lines indicate copying. Source \cite{deepmind:graphnets}}
        \label{fig:gn-enc-proc-dec}
\end{figure}
\clearpage

\subsection{Dataset Samples}

\begin{figure}[h]
    \centering
    \includegraphics[clip,width=\columnwidth]{resources/dataset/v2-000011.png}\vspace{4mm}
    \includegraphics[clip,width=\columnwidth]{resources/dataset/v2-000100.png}\vspace{4mm}
    \includegraphics[clip,width=\columnwidth]{resources/dataset/v2-001000.png}\vspace{4mm}
    \includegraphics[clip,width=\columnwidth]{resources/dataset/v2-010000.png}\vspace{4mm}
    \includegraphics[clip,width=\columnwidth]{resources/dataset/v2-050000.png}\vspace{4mm}
    \includegraphics[clip,width=\columnwidth]{resources/dataset/v2-100000.png}
    \caption[Samples from dataset version 2]{Samples from dataset version 2 (desktop and mobile screenshots). Ranks, from top to bottom: \#11; \#100; \#1,000; \#10,000; \#50,000; \#100,000.}
    \label{fig:dataset samples}
\end{figure}\clearpage

\section{Source Code}

\begin{lstlisting}[
    label={lst:getthreefold},
    language=Python,
    caption={Dynamic and deterministic dataset splitting},
    captionpos=b
]
from collections import namedtuple
from typing import List, Type

Data = namedtuple('ThreefoldData', ['train', 'valid', 'test'])

def get_threefold(klass: Type, sample_paths: List[str], train_ratio: float, valid_ratio: float, logrank_b: float) -> Data:
    """
    :param klass: Dataset class, e.g. DatasetV2
    :param sample_paths: List of paths that point to the samples of the dataset
    :param train_ratio: Value in [0,1], ratio of training samples
    :param valid_ratio: Value in [0,1], ratio of validation samples
    :param logrank_b: Logrank base (makes the weighting steeper b --> 0, more linear b --> 10, or inverted b > 10)
    :return: Three datasets (train, validation, test)
    """

    assert train_ratio + valid_ratio <= 1., "Train and validation ratio must be less than or equal to 1."
    assert len(sample_paths) > 0, "No dataset samples found."

    sample_paths = sorted(sample_paths)

    train_paths, valid_paths, test_paths = [], [], []

    for path in sample_paths:
        n_train, n_valid, n_test = len(train_paths), len(valid_paths), len(test_paths)
        n_total = n_train + n_valid + n_test

        if n_total == 0 or n_train / n_total < train_ratio:
            train_paths.append(path)
        elif n_valid / n_total < valid_ratio:
            valid_paths.append(path)
        else:
            test_paths.append(path)

    return Data(
        train=klass(train_paths, logrank_b),
        valid=klass(valid_paths, logrank_b),
        test=klass(test_paths, logrank_b))
\end{lstlisting}
\clearpage

\section{Activation Maps}

\begin{figure}[h]
    \centering
    \includegraphics[clip,width=\columnwidth]{resources/analysis/feat-map-4058-0.png}\\
    \includegraphics[clip,width=\columnwidth]{resources/analysis/feat-map-4058-1.png}\\
    \includegraphics[clip,width=\columnwidth]{resources/analysis/feat-map-4058-2.png}\\
    \includegraphics[clip,width=\columnwidth]{resources/analysis/feat-map-4058-3.png}\\
    \includegraphics[clip,width=\columnwidth]{resources/analysis/feat-map-4058-4.png}\\
    \caption[Activation maps of the CNN for sample \#4058]{Cherry-picked activation maps of the CNN for sample \#4058. Top to bottom: Input, Block1, Block2, Block3, and Block4.}
    \label{fig:activationmaps2}
\end{figure}

\begin{figure}[h]
    \centering
    \includegraphics[clip,width=\columnwidth]{resources/analysis/feat-map-16364-0.png}\\
    \includegraphics[clip,width=\columnwidth]{resources/analysis/feat-map-16364-1.png}\\
    \includegraphics[clip,width=\columnwidth]{resources/analysis/feat-map-16364-2.png}\\
    \includegraphics[clip,width=\columnwidth]{resources/analysis/feat-map-16364-3.png}\\
    \includegraphics[clip,width=\columnwidth]{resources/analysis/feat-map-16364-4.png}\\
    \caption[Activation maps of the CNN for sample \#16364]{Cherry-picked activation maps of the CNN for sample \#16364. Top to bottom: Input, Block1, Block2, Block3, and Block4.}
    \label{fig:activationmaps3}
\end{figure}
\clearpage
