\documentclass{article}
\usepackage[utf8]{inputenc}
\usepackage{tikz}

\title{Pagerank Estimation with Deep Graph Networks}
\author{Timo Denk, Samed Güner}
\date{June 2019}

% todo
\usepackage{xargs}                      % Use more than one optional parameter in a new commands
\usepackage[colorinlistoftodos,prependcaption,textsize=tiny]{todonotes}
\newcommandx{\unsure}[2][1=]{\todo[linecolor=red,backgroundcolor=red!25,bordercolor=red,#1]{#2} \PackageWarning{}{#2!}}
\newcommandx{\change}[2][1=]{\todo[linecolor=blue,backgroundcolor=blue!25,bordercolor=blue,#1]{#2}\PackageWarning{}{#2!}}
\newcommandx{\info}[2][1=]{\todo[linecolor=OliveGreen,backgroundcolor=OliveGreen!25,bordercolor=OliveGreen,#1]{#2}\PackageWarning{}{#2!}}
\newcommandx{\improve}[2][1=]{\todo[linecolor=Plum,backgroundcolor=Plum!25,bordercolor=Plum,#1]{#2}\PackageWarning{}{#2!}}
\newcommandx{\thiswillnotshow}[2][1=]{\todo[disable,#1]{#2}\PackageWarning{}{#2!}}

% citing
\usepackage[backend=bibtex,style=alphabetic]{biblatex}
\addbibresource{bibliography.bib}

\begin{document}

\maketitle

\section{Introduction}
General description, open problems, motivation, goals of this work, contributions, structure

\section{Background}
Methods, literature, related work (perhaps as a separate subsection)
Description of GCP, K8s, infrastructure background

\subsection{Ground Truth}
\subsubsection{Open PageRank}
\label{OpenPageRank}
\textit{Open PageRank} is an initiative, which has the goal to enable webmasters to check the pagerank for their domains and easily compare with other domains. The initiative was constituted after Google had shut down the \textit{PageRank}-toolbar and left a void in the industry. The Open PageRank initiative provides completely free and available data that was built on top of \textit{Common Crawl}, which provides high quality crawl data of webpages since 2013. \cite{OpenPageRank} \cite{CommonCrawl}

Open PageRank uses the number of backlinks of a domain found in \textit{Common Crawl} to calculate the pagerank. The more the number of links different domains use to refer to the given domain, the better is the pagerank of the given domain in Open PageRank.


\section{Method}
Our contributions, novel aspects, great detail

% Datasets Section

\section{Datasets}

In our approach we propose a supervised machine learning algorithm to estimate the page rank using deep graph networks. For this purpose we require labeled data to develop, train, and validate our model.

In general the process of gathering, augmenting, and labeling data can take an immense amount of time and can slow down the development of machine learning models. Therefore, we decided to develop iteratively and align our dataset to the current needs of research and progress of development. For each upcoming iteration we specified the requirements for the dataset. Following from this, each iteration resulted in a corresponding dataset version.

The following sections will profoundly specify each dataset version as required during the development phase.

\subsection{Dataset Version 1}
\label{DatasetVersion1}
The dataset version 1 consists of 100,000 samples in total. The ground truth for this dataset version is OpenPage Rank, which is described in Section \ref{OpenPageRank}.

Every sample $X$ is corresponds to a domain contained in the ground truth. We denote the $r$th sample as $X^{(r)}$. It can be expressed as a tuple consisting of four values:

\begin{center}
 $X^{(r)} = (\tensorsym{I}, r, u)$
\begin{itemize}
	\item[$\tensorsym{I}$] The screenshot of the domain in image format PNG and in the resolution $1920\times1080$. The image tensor has four channels, $\tensorsym{I}$ is therefore $\in\mathbb{R}^{1920\times1080\times4}$.
	\item[$r$] The global rank of the domain, according to Open PageRank, within $[1, 100000]$. 
	\item[$u$] The URL of the domain mapped by the Open PageRank list. For example a string like \texttt{github.com}.
\end{itemize}
\end{center}

\subsection{Dataset Version 2}
\label{DatasetVersion2}
The dataset version 2 consists of 100,000 samples in total and the ground truth is Open PageRank as described in dataset version 1.

Every sample $X$ is corresponds to a domain contained in the ground truth. We denote the $r$th sample as $X^{(r)}$. It can be expressed as a tuple consisting of four values:

\begin{center}
$X^{(r)} = (G,u,r,h)$
\begin{itemize}
    \item[$G$] The directed graph of the domain characterized by a set of nodes $\mathbb{V}$ and edges $\mathbb{E}$: $G= \left(\mathbb{V}, \mathbb{A}\right)$.
	\item[$u$] The URL of the domain mapped by the Open PageRank list.
	\item[$r$] The global rank of the domain, according to Open PageRank, within $[1, 100000]$. 
    \item[$h$] Indicator for whether or not the website uses the protocol HTTPS, the value is $\in\left\{\text{true}, \text{false}\right\}$.
\end{itemize}
\end{center}

\subsubsection{The Directed Graph}
The directed graph provides a topological view on the given website by representing all associated web pages within the same domain as vertices and connections as arrows. Figure \ref{fig:PartialDirectedGraph_timodenk.com} illustrates this by means of an example. During the crawling process as described in Section \ref{Datacrawler} vertices and arrows of the graph are created and enriched with local information about the visited web pages.

The directed graph $G$ is characterized by a set of nodes $\mathbb{V}$ and edges $\mathbb{E}$. Each node $v \in \mathbb{V}$ represents a web page of the given domain associated with the graph. A node is characterized by the following tuple:

\begin{center}
$v = (u, \tensorsym{I}, \tensorsym{M},t, l, s)$
\begin{itemize}
	\item[$u$] The exact URL of the web page. Example: \texttt{https://github.com/help}
	\item[$\tensorsym{I}$] The screenshot of web page in image format PNG and in the resolution $1920\times1080$.
	\item[$\tensorsym{M}$] The screenshot of web page in image format PNG and in mobile resolution \todo{Add value}{$w\times h$}.
	\item[$t$] The title of web page as can be seen in the tabs bar of a common browser.
	\item[$l$] The time it takes to download all data of web page from the web server in milliseconds.
	\item[$s$] Number of kilobytes downloaded in total for web page (size).
\end{itemize}
\end{center}

Each edge represents a possible navigation from web page $v_1$ to $v_2$ in the graph $G$. It can be understood as a hyperlink or button found on web page $v_1$ (source), which points to $v_2$ (target) in the same domain. 

\begin{center}
	$v \in (v_1, v_2, t)$
	\begin{itemize}
		\item[$v_1$] The source node (web page).
		\item[$v_2$] The target node that $v_1$ points to.
		\item[$t$] If existent, the text that links from source to target web page.
	\end{itemize}
\end{center}

\begin{figure}
\centering
\usetikzlibrary{shapes.multipart}
  \tikzset{
	vertices/.style = {   
		text width=12.0em, align=center,                                           
		draw,
		rectangle split,
		rectangle split parts=6
	}
}
\scalebox{.85}{
\begin{tikzpicture}
\node[vertices]  (timodenk) at (5,4) {
	\nodepart{one} \texttt{timodenk.com}
	\nodepart{two} \includegraphics[width=.5\textwidth]{resources/timodenk}
	\nodepart{three} \includegraphics[width=.25\textwidth]{resources/timodenk_mobile}
	\nodepart{four} Timo Denk
	\nodepart{five} 636 ms
	\nodepart{six} 434 kb
};

\node[vertices]  (sites) at (10,2) {
\nodepart{one} \texttt{sites.timodenk.com}
\nodepart{two} \includegraphics[width=.5\textwidth]{resources/sites_timodenk}
\nodepart{three} \includegraphics[width=.25\textwidth]{resources/sites_timodenk_mobile}	\nodepart{four} Timo Denk Sites Overview
\nodepart{five} 767 ms
\nodepart{six} 737 kb
};

\node[vertices]  (imprint) at (0,2) {
\nodepart{one} \texttt{timodenk.com/imprint}
\nodepart{two} \includegraphics[width=.5\textwidth]{resources/imprint_timodenk}
\nodepart{three} \includegraphics[width=.25\textwidth]{resources/imprint_timodenk_mobile}
\nodepart{four} Imprint - www.timodenk.com
\nodepart{five} 304 ms
\nodepart{six} 52 kb
};

\node (timodenkToImprint) at (0.5,6) {\texttt{timodenk.com/imprint}};
\node (timodenkToSites) at (9.5,6) {\texttt{sites.timodenk.com}};
\node (SitesToTimodenk) at (5.5,0.25) {\texttt{timodenk.com}};
\draw[-latex] (timodenk.one east) to [bend left=30] (sites.one north);
\draw[-latex] (sites.six west) to [bend left=30] (timodenk.six south);
\draw[-latex] (timodenk.one west) to [bend right=30](imprint.one north);
\end{tikzpicture}
}
\caption{Illustrates partial directed graph of domain \texttt{timodenk.com} with three web pages, corresponding arrows and vertices}
\label{fig:PartialDirectedGraph_timodenk.com}
\end{figure}


\section{Implementation Details}
Technical aspects of our methods, things we have tried that might have failed, code snippets
Technical description of the dataset creation process

\section{Results}
Presentation of our results, no opinion

\section{Discussion}
Discussion of the results, assumptions about why things are the way they are

\section{Conclusion}
Summary, outlook, future work

\section{References}
Literature, papers, weblinks
\printbibliography

\end{document}
